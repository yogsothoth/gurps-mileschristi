\chapter{Le personnage}\label{chap:personnage}
\section{Créer un Templier}\label{sec:creeruntemplier}
\subsection{Points de départ}\label{sec:pointsdedepart}
Les templiers sont des personnages puissants. Par conséquent, les règles
proposées ici s'écartent du coût standard de 100 points proposé par
\cite[p.~20]{B}, et offrent aux joueurs 340 points pour construire leur
personnage\index{Points de depart@Points de départ}. Afin d'éviter les débordements, le nombre maximum de points pouvant être dépensés dans les faiblesses est fixé à 50. La dualité moine-chevalier
n'implique en revanche pas de limitation quant au nombre de points à répartir
dans les compétences de l'un ou l'autre de ces deux aspects.
\subsection{Le moine}\label{sec:lemoine}
Les templiers appartiennent à un Ordre monacal, Le Temple, et sont donc à ce
titre des moines\index{Moine!introduction}. Il en découle que certaines obligations s'imposent aux
personnages, lesquelles se traduisent en termes de jeu en différents compteurs
représentant les vœux monastiques et leur satisfaction. Les moines doivent
également explorer les différents aspects de la foi et d'un mode de vie guidé
par celle-ci. Dans le jeu, cela signifie que les joueurs se doivent d'investir
des points dans une sélection de compétences liées à la religion, sa pratique
et les différentes facettes de la foi : Amour, Intelligence et Volonté. Les
points pouvant être investis dans les compétences régies par le cadre de moine
sont limitées par le score dont dispose le personnage dans l'avantage Moine
(voir la section \titleref{sec:avantages}, \vpageref{sec:avantages}).
\subsection{Le chevalier}\label{sec:lechevalier}
Les templiers sont également des chevaliers\index{Chevalier!introduction}. Ils ont à ce titre des vertus à
développer et à respecter. Dans le même esprit que pour les vœux monastiques,
un manquement à une vertu chevaleresque peut occasionner des régressions dans
les compétences associées. Le nombre de points qu'un joueur peut investir dans
ses compétences de chevalier est limité par le score du personnage dans
l'avantage Chevalier (voir la section \titleref{sec:avantages},
\vpageref{sec:avantages}).
\subsection{Le templier}\label{sec:letemplier}
L'aspect templier\index{Templier!introduction} du personnage représente la synthèse des deux composantes que
sont le Moine et le Chevalier. Un bon templier se doit d'être à la fois un bon
moine mais aussi un bon chevalier. L'un comme l'autre des deux aspects doit
être entretenu et développé par le personnage afin de s'attirer les faveurs
divines. En termes de jeu, cela se traduit par un score de Templier égal au
plus petit des deux scores en Moine et en Chevalier. Ce score permet par la suite d'identifier
la mesure des faveurs divines accordées en jeu. Celles-ci se déclinent en deux natures : d'une part, des prodiges que le frère Templier peut demander et d'autre part des relances de dés dont le tableau ci-dessous donne le nombre maximum par séance. Les prodiges, leur obtention, réalisation et leur nature sont détaillés plus loin (voir \titleref{sec:prodiges}, \vpageref{sec:prodiges}).
\index{Templier!progression du score}
\begin{center} 
\tablecaption{\label{t-progressiontemplier}Progression du score de Templier}
\tablehead
	{\hline \textbf{Score} & \textbf{Relances} \\ \hline}
\tabletail
	{\hline \multicolumn{2}{c}{\emph{Suite page suivante...}}\\}
\tablelasttail
	{\hline}
\begin{supertabular}{|c|c|}
\hline
1-5 & 1 \\ 
\rowcolor[gray]{0.8}
6-8 & 2 \\ 
9-10 & 3 \\ 
\hline
\end{supertabular}
\end{center}

%%\begin{table}[h]
%%\caption{\label{t-progressiontemplier} Progression du score de Templier}
%%\centering
%%\begin{tabular}{|c|c|}
%\hline Score & Relances \\ 
%\hline 1-5 & 1 \\ 
%\hline 6-8 & 2 \\ 
%\hline 9-10 & 3 \\ 
%\hline 
%\end{tabular}
%\end{table}
\section{Avantages}\label{sec:avantages}
Le tableau ci-dessous recense les différents avantages\index{Avantages!généraux} pertinents pour tout
personnage dans \gmc. D'autres avantages peuvent également être achetés, à
charge du Maître de Jeu d'assurer l'équilibre du jeu. Il convient de remarquer
de certains avantages comme Blessed (\cite[p.93]{R}) ou Clerical Magic
(\cite[p.35]{CI}) ne figurent pas dans cette liste, car la conception de la
religion supposée par ces avantages ne correspond pas à celle régissant une
partie de \gmc. De même, certains de ces avantages ne conviennent pas à un
Templier (ou à un Assassin, par exemple) et il s'agit pour le Maître de Jeu de
s'assurer de la cohérence de la sélection faite par les joueurs.

\begin{center} 
\tablecaption{\label{t-avantages}Avantages pour un personnage}
\tablehead
	{\hline \textbf{Avantage} & \textbf{Coût} & \textbf{Référence}\\ \hline}
\tabletail
	{\hline \multicolumn{3}{c}{\emph{Suite page suivante...}}\\}
\tablelasttail
	{\hline}
\begin{supertabular}{|c|c|c|}
\hline 
Absolute Direction & 5 & \cite[p.19]{B} \\ 
Acute Hearing & 2/level & \cite[p.19]{B} \\ 
\rowcolor[gray]{0.8}
Acute Taste and Smell & 2/level & \cite[p.19]{B} \\ 
\rowcolor[gray]{0.8}
Acute Vision & 2/level & \cite[p.19]{B} \\
Administrative Rank & 5/level & \cite[p.19]{CI} \\
Alertness & 5/level & \cite[p.19]{B} \\
Ally (Templier) & 25 & \cite[p.23]{B} \\ 
\rowcolor[gray]{0.8}
Ally (Autre) & Variable & \cite[p.23]{B} \\ 
\rowcolor[gray]{0.8}
Ambidexterity & 10 & \cite[p.19]{B} \\
Animal Empathy & 5 & \cite[p.19]{B} \\
Charisma & 5/level & \cite[p.19]{B} \\
\rowcolor[gray]{0.8}
Clerical Investment & 5/level & \cite[p.92]{R} \\
\rowcolor[gray]{0.8}
Collected & 5 & \cite[p.22]{CI} \\
Combat Reflexes & 15 & \cite[p.20]{B} \\
Common Sense & 10 & \cite[p.20]{B} \\
\rowcolor[gray]{0.8}
Composed & 5 & \cite[p.22]{CI} \\
\rowcolor[gray]{0.8}
Contacts & Variable & \cite[p.22]{CI} \\
Cultural Adaptability & 25 & \cite[p.23]{CI} \\
Danger Sense & 15 & \cite[p.20]{B} \\
\rowcolor[gray]{0.8}
Daredevil & 15 & \cite[p.23]{CI} \\
\rowcolor[gray]{0.8}
Destiny & Variable & \cite[p.35]{CI} \\
Divine Favor & Variable & \cite[p.36]{CI} \\
Eidetic Memory & 30/60 & \cite[p.20]{B} \\
\rowcolor[gray]{0.8}
Empathy & 15 & \cite[p.20]{B} \\
\rowcolor[gray]{0.8}
Faith Healing & 30 & \cite[p.36]{CI} \\
Favor & Variable & \cite[p.25]{CI} \\
Fearlessness & 2/level & \cite[p.25]{CI} \\
\rowcolor[gray]{0.8}
Fit & 5 & \cite[p.25]{CI} \\
\rowcolor[gray]{0.8}
Heir & 5 & \cite[p.25]{CI} \\
High Pain Threshold & 10 & \cite[p.20]{B} \\
Higher Purpose & 5 & \cite[p.26]{CI} \\
\rowcolor[gray]{0.8}
Imperturbable & 10 & \cite[p.26]{CI} \\
\rowcolor[gray]{0.8}
Intuition & 15 & \cite[p.20]{B} \\
Language Talent & 2/level & \cite[p.20]{B} \\
Less Sleep & 3/level & \cite[p.27]{CI} \\
\rowcolor[gray]{0.8}
Literacy & Variable & \cite[p.21]{B} \\
\rowcolor[gray]{0.8}
Patron (Temple) & 30 & \cite[p.24]{B} \\
Perfect Balance & 15 & \cite[p.63]{CI} \\
Pious & 5 & \cite[p.29]{CI} \\
\rowcolor[gray]{0.8}
Pitiable & 5 & \cite[p.29]{CI} \\
\rowcolor[gray]{0.8}
Rapid Healing & 5 & \cite[p.22]{B} \\
Reputation & Variable & \cite[p.17]{B} \\
Semi-Literacy & 0 or 5 & \cite[p.29]{CI} \\
\rowcolor[gray]{0.8}
Sensitive & 5 & \cite[p.30]{CI} \\
\rowcolor[gray]{0.8}
Status & 5/level & \cite[p.18]{B} \\
Strong Will & 4/level & \cite[p.23]{B} \\
Tenure & 5 & \cite[p.31]{CI} \\
\rowcolor[gray]{0.8}
Toughness & 10/25 & \cite[p.23]{B} \\
\rowcolor[gray]{0.8}
Versatile & 5 & \cite[p.31]{CI} \\
Very Fit & 15 & \cite[p.31]{CI} \\
Very Rapid Healing & 15 & \cite[p.31]{CI} \\
\rowcolor[gray]{0.8}
Voice & 10 & \cite[p.23]{B} \\
\rowcolor[gray]{0.8}
Wealth & Variable & \cite[p.16]{B} \\
\hline 

%%\end{tabular}
\end{supertabular}
\end{center}

\section{Nouveaux avantages}\label{sec:nouveauxavantages}
\index{Avantages!nouveaux}
\subsection{Moine}\label{subsec:av-moine}
L'avantage Moine\index{Moine!avantage} permet à la fois de synthétiser les qualités d'un personnage
dans le domaine du sacré, ainsi que d'en identifier les limites. Du score de
Moine dépendent les score maximum des compétences qui y sont rattachées. Le
fonctionnement de cet avantage reprend celui nommé \og Arete \fg{} et décrit dans
\cite[p.44]{MtA}. Il est important de noter que cet avantage peut progresser par la suite, de la même manière que l'avantage nommé \og Arete \fg{} et décrit dans \gurpstitle{Mage: The Ascension} (voir \cite[p.132]{MtA}). Le tableau ci-dessous en décrit les coûts et les limites
imposées aux compétences de moine (voir la section
\titleref{subsec:comp-moine}, \vpageref{subsec:comp-moine}).
\index{Moine!progression du score}
\begin{center} 
\tablecaption{\label{t-progressionmoine}Progression du score de Moine}
\tablehead
	{\hline \textbf{Score de Moine} & \textbf{Max.\ compétence} & \textbf{Coût}\\ \hline}
\tabletail
	{\hline \multicolumn{3}{c}{\emph{Suite page suivante...}}\\}
\tablelasttail
	{\hline}
\begin{supertabular}{|c|c|c|}
\hline 
1 & 11  & 15 points \\ 
2 & 12  & 30 points \\ 
\rowcolor[gray]{0.8}
3 & 13 & 45 points \\ 
\rowcolor[gray]{0.8}
4 & 14 & 65 points \\
5 & 15 & 85 points \\ 
6 & 16 & 110 points \\
\rowcolor[gray]{0.8}
7 & 17 & 135 points \\
\rowcolor[gray]{0.8}
8 & 18 & 170 points \\ 
9 & 19 & 200 points \\ 
10 & Aucune limite & 250 points \\  
\hline
\end{supertabular}
\end{center}


%\begin{table}[h]
%\caption{\label{t-progressionmoine} Progression du score de Moine}
%\centering
%\begin{tabular}{|c|c|c|}
%\hline Moine & Max.\  compétence & Coût \\ 
%\hline 1 & 11 & 15 points \\ 
%\hline 2 & 12 & 30 points \\ 
%\hline 3 & 13 & 45 points \\ 
%\hline 4 & 14 & 65 points \\ 
%\hline 5 & 15 & 85 points \\ 
%\hline 6 & 16 & 110 points \\ 
%\hline 7 & 17 & 135 points \\ 
%\hline 8 & 18 & 170 points \\ 
%\hline 9 & 19 & 200 points \\ 
%\hline 10 & Aucune limite & 250 points \\  
%\hline 
%\end{tabular}
%\end{table}
\subsection{Chevalier}\label{subsec:av-chevalier}
L'avantage Chevalier\index{Chevalier!avantage} permet à la fois de synthétiser les qualités d'un
personnage en tant que chevalier, ainsi que d'en identifier les limites. Du
score de Chevalier dépendent les score maximum des compétences qui y sont
rattachées. Le fonctionnement de cet avantage reprend celui nommé \og Arete \fg{} et
décrit dans \cite[p.44]{MtA}. Il est important de noter que cet avantage peut progresser par la suite, de la même manière que l'avantage nommé \og Arete \fg{} et décrit dans \gurpstitle{Mage: The Ascension} (voir \cite[p.132]{MtA}). Le tableau ci-dessous en décrit les coûts et les
limites imposées aux compétences de chevalier (voir la section
\titleref{subsec:comp-chevalier}, \vpageref{subsec:comp-chevalier}).
\index{Chevalier!progression du score}
\begin{center} 
\tablecaption{\label{t-progressionchevalier} Progression du score de Chevalier}
\tablehead
	{\hline \textbf{Score de Chevalier} & \textbf{Max.\ compétence} & \textbf{Coût}\\ \hline}
\tabletail
	{\hline \multicolumn{3}{c}{\emph{Suite page suivante...}}\\}
\tablelasttail
	{\hline}
\begin{supertabular}{|c|c|c|}
\hline 
1 & 11  & 15 points \\ 
2 & 12  & 30 points \\ 
\rowcolor[gray]{0.8}
3 & 13 & 45 points \\ 
\rowcolor[gray]{0.8}
4 & 14 & 65 points \\
5 & 15 & 85 points \\ 
6 & 16 & 110 points \\
\rowcolor[gray]{0.8}
7 & 17 & 135 points \\
\rowcolor[gray]{0.8}
8 & 18 & 170 points \\ 
9 & 19 & 200 points \\ 
10 & Aucune limite & 250 points \\  
\hline
\end{supertabular}
\end{center}


%\begin{table}[h]
%\caption{\label{t-progressionchevalier} Progression du score de Chevalier}
%\centering
%\begin{tabular}{|c|c|c|}
%\hline Chevalier & Max.\  compétence & Coût \\ 
%\hline 1 & 11 & 15 points \\ 
%\hline 2 & 12 & 30 points \\ 
%\hline 3 & 13 & 45 points \\ 
%\hline 4 & 14 & 65 points \\ 
%\hline 5 & 15 & 85 points \\ 
%\hline 6 & 16 & 110 points \\ 
%\hline 7 & 17 & 135 points \\ 
%\hline 8 & 18 & 170 points \\ 
%\hline 9 & 19 & 200 points \\ 
%\hline 10 & Aucune limite & 250 points \\  
%\hline 
%\end{tabular}
%\end{table}
\section{Faiblesses}\label{sec:faiblesses}
\index{Faiblesses}
Le tableau ci-dessous recense les différentes faiblesses pertinentes pour tout
type de personnage dans \gmc. D'autres avantages peuvent également être achetés,
à charge du Maître de Jeu d'assurer l'équilibre du jeu. Il convient de remarquer
de certaines avantages comme Blessed (\cite[p.93]{R}) ou Clerical Magic
(\cite[p.35]{CI}) ne figurent pas dans cette liste, car la conception de la
religion supposée par ces avantages ne correspond pas à celle régissant une
partie de \gmc. De même, certaines de ces faiblesses ne conviennent pas à un
Templier (ou à un Assassin, par exemple) et il s'agit pour le Maître de Jeu de
s'assurer de la cohérence de la sélection faite par les joueurs.

\begin{center} 
\tablecaption{\label{t-faiblesses}Faiblesses pour un personnage}
\tablehead
	{\hline \textbf{Faiblesse} & \textbf{Coût} & \textbf{Référence}\\ \hline}
\tabletail
	{\hline \multicolumn{3}{c}{\emph{Suite page suivante...}}\\}
\tablelasttail
	{\hline}
\begin{supertabular}{|c|c|c|}
Absent Mindedness & -15 & \cite[p.30]{B} \\ 
Addiction & Variable & \cite[p.30]{B} \\ 
\rowcolor[gray]{0.8}
Age & -3/year & \cite[p.27]{B} \\ 
\rowcolor[gray]{0.8}
Albinism & -10 & \cite[p.27]{B} \\ 
Alcohol Intolerance & -1 & \cite[p.79]{CI} \\ 
Alcoholism & -15/-30 & \cite[p.30]{B} \\ 
\rowcolor[gray]{0.8}
Alcohol-Related Quirks & -1 & \cite[p.79]{CI} \\ 
\rowcolor[gray]{0.8}
Allergic Susceptibility & -5 à -15 & \cite[p.96]{CI} \\ 
Amnesia & -10/-25 & \cite[p.86]{B} \\ 
Anaerobic & -30 & \cite[p.101]{CI} \\ 
\rowcolor[gray]{0.8}
Appearance & Variable & \cite[p.15]{B} \\ 
\rowcolor[gray]{0.8}
Attentive & -1 & \cite[p.86]{CI} \\ 
Bad Back & -15/-25 & \cite[p.80]{CI} \\ 
Bad Sight & -10/-25 & \cite[p.27]{B} \\ 
\rowcolor[gray]{0.8}
Bad Smell & -10 & \cite[p.80]{CI} \\ 
\rowcolor[gray]{0.8}
Bad Temper & -10 & \cite[p.31]{B} \\ 
Blindness & -50 & \cite[p.27]{B} \\ 
Bloodlust & -10 & \cite[p.31]{B} \\ 
\rowcolor[gray]{0.8}
Bloodthirst & -15 & \cite[p.96]{CI} \\ 
\rowcolor[gray]{0.8}
Bowlegged & -1 & \cite[p.80]{CI} \\ 
Broad-Minded & -1 & \cite[p.86]{CI} \\ 
Bully & -10 & \cite[31]{B} \\
\rowcolor[gray]{0.8}
Callous & -6 & \cite[86]{CI} \\
\rowcolor[gray]{0.8}
Careful & -1 & \cite[86]{CI} \\
Chauvinistic & -1 & \cite[87]{CI} \\
Chronic Depression & Variable & \cite[87]{CI} \\
\rowcolor[gray]{0.8}
Chummy & -5 & \cite[87]{CI} \\
\rowcolor[gray]{0.8}
Clueless & -10 & \cite[87]{CI} \\
Cold-Blooded & -5/-10 & \cite[p.101]{CI} \\
Colour Blindness & -10 & \cite[28]{B} \\
\rowcolor[gray]{0.8}
Combat Paralysis & -15 & \cite[32]{B} \\
\rowcolor[gray]{0.8}
Compulsive Behaviour & -5 à -15 & \cite[32]{B} \\
Compulsive Lying & -15 & \cite[32]{B} \\
Confused & -10 & \cite[88]{CI} \\
\rowcolor[gray]{0.8}
Congenial & -1 & \cite[89]{CI} \\
\rowcolor[gray]{0.8}
Cowardice & -10 & \cite[32]{B} \\
Curious & -5 à -15 & \cite[89]{CI} \\
Deafness & -20 & \cite[28]{B} \\
\rowcolor[gray]{0.8}
Delicate Metabolism & -20/-40 & \cite[81]{B} \\
\rowcolor[gray]{0.8}
Delusions & -1 à -15 & \cite[32]{B} \\
Dependency & Variable & \cite[81]{CI} \\
Destiny & Variable & \cite[97]{CI} \\
\rowcolor[gray]{0.8}
Distractible & -1 & \cite[89]{CI} \\
\rowcolor[gray]{0.8}
Dreamer & -1 & \cite[89]{CI} \\
Dull & -1 & \cite[89]{CI} \\
Dyslexia & -5 à -15 & \cite[33]{B} \\
\rowcolor[gray]{0.8}
Easy To Read & -10 & \cite[89]{CI} \\
\rowcolor[gray]{0.8}
Ennemy & Variable & \cite[39]{B} \\
Ennemy (Inconnu) & Variable & \cite[77]{CI} \\
Epilepsy & -30 & \cite[28]{B} \\
\rowcolor[gray]{0.8}
Evil Twin & Variable & \cite[77]{CI} \\
\rowcolor[gray]{0.8}
Extra Sleep & -3/level & \cite[81]{CI} \\
Extreme Fanaticism & -15 & \cite[90]{CI} \\
Fanaticism & -15 & \cite[33]{B} \\
\rowcolor[gray]{0.8}
Fat & -10/-20 & \cite[28]{B} \\
\rowcolor[gray]{0.8}
Flashbacks & -5 à -20 & \cite[90]{CI} \\
Fragile & -20 & \cite[102]{CI} \\
Glory Hound & -15 & \cite[90]{CI} \\
\rowcolor[gray]{0.8}
Gluttony & -5 & \cite[33]{B} \\
\rowcolor[gray]{0.8}
Greed & -15 & \cite[33]{B} \\
Gregarious & -10 & \cite[90]{CI} \\
Guilt Complex & -5 & \cite[90]{CI} \\
\rowcolor[gray]{0.8}
Gullibility & -10 & \cite[33]{B} \\
\rowcolor[gray]{0.8}
Hard Of Hearing & -10 & \cite[28]{B} \\
Hemophilia & -30 & \cite[28]{B} \\
Hideboud & -5 & \cite[91]{CI} \\
\rowcolor[gray]{0.8}
Honesty & -10 & \cite[33]{B} \\
\rowcolor[gray]{0.8}
Horrible Hangovers & -1 & \cite[79]{CI} \\
Humble & -1 & \cite[91]{CI} \\
Ignorance & -5/skill & \cite[78]{CI} \\
\rowcolor[gray]{0.8}
Illiteracy & -10 & \cite[33]{B} \\
\rowcolor[gray]{0.8}
Imaginative & -1 & \cite[91]{CI} \\
Impulsiveness & -10 & \cite[33]{B} \\
Incompetence & -1 & \cite[91]{CI} \\
\rowcolor[gray]{0.8}
Incurious & -5 & \cite[91]{CI} \\
\rowcolor[gray]{0.8}
Indecisive & -10 & \cite[91]{CI} \\
Innumerate & -1/-5/-10 & \cite[91]{CI} \\
Insomniac & -10/-15 & \cite[82]{CI} \\
\rowcolor[gray]{0.8}
Intolerance & Variable & \cite[34]{B} \\
\rowcolor[gray]{0.8}
Jealousy & -10 & \cite[34]{B} \\
Jinxed & -20 à -60 & \cite[102]{CI} \\
Killjoy & -15 & \cite[91]{CI} \\
\rowcolor[gray]{0.8}
Kleptomania & -15 & \cite[34]{B} \\
\rowcolor[gray]{0.8}
Klutz & -5/-15 & \cite[82]{CI} \\
Lame & -15 à -35 & \cite[34]{B} \\
Laziness & -10 & \cite[34]{B} \\
\rowcolor[gray]{0.8}
Lecherousness & -15 & \cite[34]{B} \\
\rowcolor[gray]{0.8}
Light Sleeper & -5 & \cite[82]{CI} \\
Loner & -5 & \cite[91]{CI} \\
Low Empathy & -15 & \cite[91]{CI} \\
\rowcolor[gray]{0.8}
Low Pain Threshold & -10 & \cite[29]{B} \\
\rowcolor[gray]{0.8}
Low Self Image & -10 & \cite[92]{CI} \\
Lunacy & -10 & \cite[92]{CI} \\
Manic Depressive & -20 & \cite[92]{CI} \\
\rowcolor[gray]{0.8}
Megalomania & -10 & \cite[34]{B} \\
\rowcolor[gray]{0.8}
Migraine & -5 à -20 & \cite[82]{CI} \\
Miserliness & -10 & \cite[34]{B} \\
Missing Digit & -2/-5 & \cite[82]{CI} \\
\rowcolor[gray]{0.8}
Mistaken Identity & -5 & \cite[78]{CI} \\
\rowcolor[gray]{0.8}
Mute & -25 & \cite[29]{B} \\
Nervous Stomach & -1 & \cite[79]{CI} \\
Night Blindness & -10 & \cite[82]{CI} \\
\rowcolor[gray]{0.8}
Nightmares & -5 & \cite[92]{CI} \\
\rowcolor[gray]{0.8}
No Sense Of Humor & -10 & \cite[92]{CI} \\
No Sense Of Smell/Tase & -5 & \cite[29]{B} \\
Nosy & -1 & \cite[92]{CI} \\
\rowcolor[gray]{0.8}
Obdurate & -10 & \cite[92]{CI} \\
\rowcolor[gray]{0.8}
Oblivious & -3 & \cite[92]{CI} \\
Obnoxious Drunk & -1 & \cite[80]{CI} \\
Obsession & -5 à -15 & \cite[93]{CI} \\
\rowcolor[gray]{0.8}
Odious Personal Habits & -5 à -15 & \cite[26]{B} \\
\rowcolor[gray]{0.8}
On The Edge & -15 & \cite[93]{CI} \\
One Arm & -20 & \cite[29]{B} \\
One Eye & -15 & \cite[29]{B} \\
\rowcolor[gray]{0.8}
One Hand & -15 & \cite[29]{B} \\
\rowcolor[gray]{0.8}
Overconfidence & -10 & \cite[34]{B} \\
Overweight & -5 & \cite[29]{B} \\
Paranoia & -10 & \cite[35]{B} \\
\rowcolor[gray]{0.8}
Personality Change & -1 & \cite[80]{CI} \\
\rowcolor[gray]{0.8}
Phobias & Variable & \cite[35]{B}, \cite[93]{CI} \\
Post-Combat Shakes & -5 & \cite[93]{CI} \\
Proud & -1 & \cite[93]{CI} \\
\rowcolor[gray]{0.8}
Pyromania & -5 & \cite[36]{B} \\
\rowcolor[gray]{0.8}
Quirks & -1 & \cite[41]{B} \\
Reduced Manual Dexterity & -3/level & \cite[83]{CI} \\
Reduced Move & -5/level & \cite[103]{CI} \\
\rowcolor[gray]{0.8}
Reputation & Variable & \cite[17]{B} \\
\rowcolor[gray]{0.8}
Responsive & -1 & \cite[93]{CI} \\
Sadism & -15 & \cite[36]{B} \\
Secret & Variable & \cite[78]{CI} \\
\rowcolor[gray]{0.8}
Secret Identity & Variable & \cite[79]{CI} \\
\rowcolor[gray]{0.8}
Self-Centered & -10 & \cite[94]{CI} \\
Self-Destruct & -20 & \cite[104]{CI} \\
Selfish & -5 & \cite[94]{CI} \\
\rowcolor[gray]{0.8}
Semi-Literacy & -5 & \cite[94]{CI} \\
\rowcolor[gray]{0.8}
Short Attention Span & -10/level & \cite[104]{CI} \\
Shyness & -5 à -15 & \cite[37]{B} \\
Skinny & -5 & \cite[29]{B} \\
\rowcolor[gray]{0.8}
Sleepwalker & -5 & \cite[84]{CI} \\
\rowcolor[gray]{0.8}
Sleepy Drinker & -1 & \cite[80]{CI} \\
Sleepy & Variable & \cite[104]{CI} \\
Social Disease & -5 & \cite[84]{CI} \\
\rowcolor[gray]{0.8}
Social Stigma & -5 à -20 & \cite[27]{B} \\
\rowcolor[gray]{0.8}
Split Personality & -10/-15 & \cite[37]{B} \\
Staid & -1 & \cite[94]{CI} \\
Status & -5 & \cite[18]{B} \\
\rowcolor[gray]{0.8}
Stress Atavism & Variable & \cite[105]{CI} \\
\rowcolor[gray]{0.8}
Stubborness & -5 & \cite[37]{B} \\
Stuttering & -10 & \cite[29]{B} \\
Supersensitive & -2/level & \cite[99]{CI} \\
\rowcolor[gray]{0.8}
Susceptibility To Poison & -5 & \cite[84]{CI} \\
\rowcolor[gray]{0.8}
Tourette's Syndrome & Variable & \cite[85]{CI} \\
Trickster & -15 & \cite[94]{CI} \\
Truthfulness & -5 & \cite[37]{B} \\
\rowcolor[gray]{0.8}
Uncongenial & -1 & \cite[94]{CI} \\
\rowcolor[gray]{0.8}
Undiscriminating & -1 & \cite[94]{CI} \\
Uneducated & -5 & \cite[79]{CI} \\
Unfit & -5 & \cite[85]{CI} \\
\rowcolor[gray]{0.8}
Unluckiness & -10 & \cite[37]{B} \\
\rowcolor[gray]{0.8}
Very Unfit & -15 & \cite[85]{CI} \\
Voices & -5 à -15 & \cite[94]{CI} \\
Vow & Variable & [37]{B} \\
\rowcolor[gray]{0.8}
Weak Will & -8/level & \cite[37]{B} \\
\rowcolor[gray]{0.8}
Workaholic & -5 & \cite[95]{CI} \\
Xenophilia & -5/-15 & \cite[95]{CI} \\
Xenophobia & -15 & \cite[36]{B} \\
\rowcolor[gray]{0.8}
Youth & -2/level & \cite[29]{B} \\
\rowcolor[gray]{0.8}


\end{supertabular}
\end{center}

\section{Echelle de la Foi}\label{sec:echellefoi}
\index{Echelle de la Foi}
L'échelle de la Foi mesure à tout moment la fidélité d'un frère à ses engagements pris envers Dieu. Elle est représentée par une série de dix cases. A l'issue d'une bonne action\index{Bonne action}, un certain nombre de cases sont cochées, en partant de la gauche : 1 pour une bonne action mineure, 3 pour une bonne action majeure. A l'inverse, si le frère cède au péché\index{Peche@Péché}, il devra abandonner des cases : 1 pour un péché mineur, 3 pour un péché grave et la totalité des cases pour un péché mortel. Les cases de péchés sont noircies en partant de la droite. Le Maître de Jeu peut également décider qu'un péché particulièrement grave peut occasionner la perte définitive d'une case sur l'échelle de la Foi. Enfin, si l'échelle compte un total de dix cases, seules un nombre de cases égal au score de Templier du frère sont utilisables. Les cases cochées pour de bonnes actions ouvrent droit à la réalisation de prodiges\index{Prodiges} par le frère (voir la section \titleref{sec:prodiges}, \vpageref{sec:prodiges}), et les cases noircies par le péché augurent d'une punition infligée au frère. La punition, la prière et la confession sont les seuls moyens de libérer les cases de la Foi du péché (voir la section \titleref{sec:peches}, \vpageref{sec:peches}). Une case noircie par le péché ne peut être cochée suite à une bonne action, mais l'inverse est vrai : une case déjà cochée suite à une bonne action peut perdre son caractère vertueux et être consomée par un péché.

\begin{quotation}
\index{Echelle de la Foi!exemple}
Exemple : Aymard est un frère dont le score de Templier est de 5. Il va donc utiliser 5 cases sur les 10 de l'échelle de la Foi. Par simplicité, nous allons numéroter ces cases de 1 à 5, en partant de la gauche. Lorsque le frère réalise une bonne action, il coche la case 1. Elle lui permet de réaliser un prodige selon les règles décrites dans la section \titleref{sec:prodiges}. Une bonne action majeure lui permet de cocher trois autres cases, les cases 2, 3 et 4. Ce frère commet ensuite un péché et noircit la case 5. Aymard réalise ensuite une autre bonne action, mais il ne peut cocher la case 5 : elle est déjà noircie par le péché. Incorrigible, Aymard cède enfin une nouvelle fois au péché. La case 4 est déjà cochée suite à l'accomplissement d'une bonne action, mais cela n'empêche pas son péché de venir entacher sa Foi : il doit noircir la case 4, perdant les fruits d'une bonne action.
\end{quotation}

\section{Compétences}\label{sec:competences}
\index{Competences@Compétences}
Les deux facettes du Templier, le Moine et le Chevalier, gouvernent chacune
un ensemble de compétences et en déterminent le niveau de maîtrise maximal,
comme décrit \vpageref{subsec:av-moine} et \vpageref{subsec:av-chevalier}.
Toutes les compétences ne relevant pas de la fonction sacrée de moine ou ne
répondant pas aux impératifs du chevalier s'identifient comme appartenant
au Siècle. Les domaines de compétences séculiers ne revêtent aucun
caractère particulier et chacun a le loisir de les développer. Un Templier
doit cependant s'assurer qu'il n'explore pas de voie contraire à celles
imposées par ses fonctions.
    
    \subsection{Compétences de moine}\label{subsec:comp-moine}
    Les compétences sacrées sont celles du Moine\index{Competences@Compétences!de moine}. Les paragraphes suivants en
    livrent une description succincte.
    \subsubsection*{Le Livre, Mental Hard, Defaults to IQ-6}
    Connaissance du Livre Sacré.
    \subsubsection*{Latin, Mental Average, No Default}
    Connaissance du Latin écrit et parlé.
    \subsubsection*{Démonologie, Mental Average, Defaults to IQ-6}
    Connaissance du Malin : comment le repérer, le détruire ainsi qu'une
    connaissance de ses agissements passés.
    \subsubsection*{Eglise, Mental Average, Defaults to IQ-6}
    Connaissance de tout ce qui a trait à la religion chrétienne :
    organisation, pouvoirs et privilèges des clercs, abbés et prélats.
    \subsubsection*{La Maison, Mental Average, Defaults to IQ-6}
    Connaissance du fonctionnement du Temple.
    \subsubsection*{Lire et écrire une langue, Mental Average, No Default}
    Capacité à lire et écrire une langue donnée. Cette compétence doit
    être achetée pour chaque langue choisie.
    \subsubsection*{Mémoire, Mental Hard, Defaults to IQ-6}
    Capacité à mémoriser les événements, les visages, les conversations ainsi
    que les passages du Livre, ainsi qu'à se les remémorer. L'écrit restant
    rare à cette époque, une mémoire bien entraînée reste intéressante.
    \subsubsection*{Médecine, Mental Hard, Defaults to IQ-7}
    Connaissances des maladies et des remèdes.
    \subsubsection*{Premiers Soins, Mental Easy, Defaults to Médecine, IQ-5}
    Soigner les blessures temporairement, poser des garrots.

    \subsection*{Compétences de chevalier}\label{subsec:comp-chevalier}
    Les compétences d'armes\index{Competences@Compétences!de chevalier}, soit sur le champ de bataille, soit en société,
    appartiennent au domaine du chevalier.
    \subsubsection*{Epée, Physical Easy, Defaults to DX-5}
    Maniement de l'épee.
    \subsubsection*{Masse Turquoise, Physical Average, Defaults to DX-5}
    Maniement de la masse turquoise.
    \subsubsection*{Couteau d'Armes, Physical Easy, Defaults to DX-4}
    Maniement du couteau d'armes.
    \subsubsection*{Ecu, Physical Easy, Defaults to DX-4}
    Maniement de l'écu.
    \subsubsection*{Lance, Physical Average, Defaults to DX-6}
    Maniement de la lance de cavalerie.
    \subsubsection*{Arc, Physical Hard, Defaults to DX-6}
    Maniement de l'arc (en société).
    \subsubsection*{Arbalète, Physical easy, Defaults to DX-4}
    Maniement de l'arbalète (en société).
    \subsubsection*{Equitation, Physical Average, Defaults to DX-5}
    Capacité à monter à cheval, à guider sa monture et à la contrôler en toute
    situation, y compris pendant les combats.
    \subsubsection*{Chasse, Mental Average, Defaults to IQ-5}
    Repérage, rabbatement et chasse proprement dite d'un gibier.
    \subsubsection*{Lutte, Physical Easy, No Default}
    Lutte à mains nues.
    \subsubsection*{Conroi, Mental Average, Defaults to IQ-6}
    Capacité à participer à une attaque de manière ordonnée, en conroi, avec
    d'autres Templiers.

    \subsection{Compétence du Siècle}\label{subsec:comp-siecle}
    Toutes les autres compétences\index{Competence@Compétences!de siècle}, ne tombant sous la coupe ni des prérogatives
    du moine, ni du savoir-faire du chevalier, appartiennent au siècle. Les
    compétences proposées ci-dessous désignent les plus courantes, les joueurs
    peuvent naturellement en sélectionner d'autres.
    \subsubsection*{Géographie, Mental Easy, No Default}
    Connaissance d'une région donnée.
    \subsubsection*{Cité, Mental Easy, No Default}
    Connaissance détaillée d'une cité particulière.
    \subsubsection{Négoce, Mental Average, Defaults to IQ-5}
    Capacité à négocier, connaissance des prix, des coutumes des marchands.
    \subsubsection*{Discourir, Mental Average, Defaults to IQ-5}
    S'exprimer en public, adapter son discours au public.
    \subsubsection*{Us et coutumes, Mental Easy, No Default}
    Connaissance des us et coutumes d'une population donnée.

    \subsection{L'Âme}\label{subsec:ame}
    L'Âme\index{Ame@Âme} représente la pureté d'un individu et se compose de trois fragments\index{Ame@Âme!fragments} :
    l'Amour\index{Amour}\index{Ame@Âme!fragment d'Amour}, la Volonté\index{Volonte@Volonté}\index{Ame@Âme!fragment de Volonté} et l'Intelligence\index{Intelligence}\index{Ame@Âme!fragment d'Intelligence}. Il convient de noter que
    l'Intelligence de l'Âme n'a pas de rapport avec l'intellect représenté par
    la caractéristique IQ, il s'agit d'une compréhension plus mystique et
    profonde, plus intuitive.
    Plus l'âme d'un individu est pure, plus la communion avec Dieu est
    facilitée et plus Sa Volonté a de chances de se manifester à travers ce
    véhicule. En termes de jeu, chaque fragment d'une âme se voit attribuer un
    score, lequel autorise la réalisation de prodiges.
    Chaque point de fragment\index{Ame@Âme!coût des fragments} coûte 5 points de personnage. Les scores de fragments ne peuvent être supérieur au score de Templier d'un personnage.
\section{Les prodiges}\label{sec:prodiges}
\index{Prodiges}
    A travers leur mode de vie particulier, cherchant à concilier chair et
    esprit, les Templiers sont amenés peut-être plus que d'autres à purifier
    leur âme. Parce que leurs actes exercent directement la Volonté de Dieu,
    leur bras et leurs mots peuvent plus facilement être guidés par Dieu. Ainsi, dans les moments de désespoir extrême, quand l'homme ne peut plus rien, le Templier peut requérir l'aide du Seigneur : il se battra comme un lion, sans crainte des coups ennemis ; il saura illuminer l'esprit de son auditoire par son éloquence parfaite ; il punira par le feu ou la foudre les ennemis de Dieu.
    En termes de jeu, le score de Templier du frère détermine le nombre maximal de prodiges\index{Prodiges!nombre maximal} qu'il peut réaliser lors d'une partie. Avant de pouvoir réaliser un prodige, un frère doit au préalable réaliser de bonnes actions\index{Bonne action}. A chaque bonne action effectuée, un certain nombre de cases sont cochées sur l'échelle de la Foi\index{Echelle de la Foi}. Ces cases indiquent le nombre de prodiges réalisables par le frère\index{Prodiges!nombre réalisable}, à tout moment. Lorsqu'un prodige est réalisé, la case est libérée. Il est important de noter que les péchés\index{Peches@Péchés} jouent en défaveur du frère, en consommant des cases sur l'échelle de la Foi. Pour plus d'informations quant à l'échelle de la Foi, reportez-vous à la section \titleref{sec:echellefoi}, \vpageref{sec:echellefoi}. Les points investis dans les différents fragments\index{Ame@Âme!fragments} (Amour, Volonté et Intelligence) indiquent quant à eux l'ampleur des prodiges\index{Prodiges!ampleur} qui peuvent être réalisés. Les prodiges sont classés selon ces trois fragments et selon le score de fragment nécessaire pour leur réalisation.
    \subsection{Prodiges d'Amour, combattre le Mal}\label{subsec:prodigesamour}
    \index{Prodiges!d'Amour}
    La table ci-dessous récapitule les prodiges présentés dans le livre de base de \gurpstitle{miles Christi}. Pour chaque prodige est indiqué le score en Amour nécessaire ainsi que la page de référence où se trouve sa description.
    \begin{center} 
\tablecaption{\label{t-prodigesamour}Prodiges d'Amour}
\tablehead
	{\hline \textbf{Prodige} & \textbf{Score} & \textbf{Référence}\\ \hline}
\tabletail
	{\hline \multicolumn{3}{c}{\emph{Suite page suivante...}}\\}
\tablelasttail
	{\hline}
\begin{supertabular}{|c|c|c|}
L'eau de Baptême & 1 & \cite[p.241]{MC} \\ 
Il nourrit cinq mille hommes & 1 & \cite[p.241]{MC} \\ 
\rowcolor[gray]{0.8}
Il guérit beaucoup de malades & 1 & \cite[p.241]{MC} \\ 
\rowcolor[gray]{0.8}
Le seigneur des eaux fécondantes & 1 & \cite[p.241]{MC} \\ 
Sym Pathos, partager la douleur & 2 & \cite[p.241]{MC} \\
Que le loup soit comme l'agneau & 2 & \cite[p.242]{MC} \\
\rowcolor[gray]{0.8}
Dieu est amour & 2 & \cite[p.242]{MC} \\ 
\rowcolor[gray]{0.8}
Agnus Dei & 3 & \cite[p.242]{MC} \\ 
Croissez et multipliez & 3 & \cite[p.242]{MC} \\
Le prodige de Génésareth & 3 & \cite[p.242]{MC} \\
\rowcolor[gray]{0.8}
L'eau de la restauration & 4 & \cite[p.242]{MC} \\ 
\rowcolor[gray]{0.8}
Ego te absoluo & 4 & \cite[p.242]{MC} \\ 
Le regard des anges & 4 & \cite[p.243]{MC} \\
Pax in nomine Domini & 5 & \cite[p.243]{MC} \\
\rowcolor[gray]{0.8}
Le baiser de Saint Lazare & 5 & \cite[p.243]{MC} \\ 
\rowcolor[gray]{0.8}
La manne céleste & 5 & \cite[p.243]{MC} \\ 
Aimez-vous les uns les autres & 5 & \cite[p.243]{MC} \\
Benedicamus Domino & 6 & \cite[p.244]{MC} \\
\rowcolor[gray]{0.8}
La trève de Dieu & 6 & \cite[p.244]{MC} \\ 
\rowcolor[gray]{0.8}
Le bouclier de la foi & 6 & \cite[p.244]{MC} \\ 
Délivre-nous du Mal & 6 & \cite[p.244]{MC} \\
Sur cette pierre je bâtirai mon église & 7 & \cite[p.244]{MC} \\
\rowcolor[gray]{0.8}
Lève-toi et marche & 7 & \cite[p.244]{MC} \\ 
\rowcolor[gray]{0.8}
Vade retro Satanas & 7 & \cite[p.244]{MC} \\ 

\end{supertabular}
\end{center}

    \subsection{Prodiges de Volonté, modifier le Monde}\label{subsec:prodigesvolonte}
    \index{Prodiges!de Volonté}
    La table ci-dessous récapitule les prodiges présentés dans le livre de base de \gurpstitle{miles Christi}. Pour chaque prodige est indiqué le score en Amour nécessaire ainsi que la page de référence où se trouve sa description.
    \begin{center} 
\tablecaption{\label{t-prodigesvolonte}Prodiges de Volonté}
\tablehead
	{\hline \textbf{Prodige} & \textbf{Score} & \textbf{Référence}\\ \hline}
\tabletail
	{\hline \multicolumn{3}{c}{\emph{Suite page suivante...}}\\}
\tablelasttail
	{\hline}
\begin{supertabular}{|c|c|c|}

Je suis la lumière du monde & 1 & \cite[p.237]{MC} \\
Donne-moi la force & 1 & \cite[p.237]{MC} \\
\rowcolor[gray]{0.8}
L'épée de sa bouche & 1 & \cite[p.237]{MC} \\ 
\rowcolor[gray]{0.8}
Les foudres du Très-Haut & 2 & \cite[p.237]{MC} \\ 
Marcher sur les eaux & 2 & \cite[p.237]{MC} \\
L'oeil de Dieu & 2 & \cite[p.237]{MC} \\
\rowcolor[gray]{0.8}
La barrière d'Osée & 2 & \cite[p.228]{MC} \\ 
\rowcolor[gray]{0.8}
Le sang de l'alliance & 3 & \cite[p.238]{MC} \\     
L'aveuglement d'Israël & 3 & \cite[p.238]{MC} \\
La gazelle & 3 & \cite[p.238]{MC} \\
\rowcolor[gray]{0.8}
La pluie de Béthoron & 3 & \cite[p.238]{MC} \\ 
\rowcolor[gray]{0.8}
Que les Ténèbres s'abattent sur ceux qui vivent dans les Ténèbres & 4 & \cite[p.238]{MC} \\ 
Pétrifier la langue & 4 & \cite[p.238]{MC} \\
L'exode & 4 & \cite[p.238]{MC} \\
\rowcolor[gray]{0.8}
La cinquième trompette & 5 & \cite[p.239]{MC} \\ 
\rowcolor[gray]{0.8}
Qui a vécu par l'épée périra par l'épée & 5 & \cite[p.239]{MC} \\ 
Le bras vengeur & 5 & \cite[p.239]{MC} \\
Le vent de l'existence précaire & 5 & \cite[p.239]{MC} \\
\rowcolor[gray]{0.8}
Le cheval invincible & 6 & \cite[p.239]{MC} \\ 
\rowcolor[gray]{0.8}
La foi soulève les montagnes & 6 & \cite[p.239]{MC} \\ 
Le déluge & 6 & \cite[p.239]{MC} \\
La femme de Lot & 6 & \cite[p.240]{MC} \\
\rowcolor[gray]{0.8}
La première trompette & 7 & \cite[p.240]{MC} \\ 
\rowcolor[gray]{0.8}
Ce que les hommes ont élevé, Dieu peut l'abaisser & 7 & \cite[p.240]{MC} \\ 
L'archange & 7 & \cite[p.240]{MC} \\

\end{supertabular}
\end{center}

    \subsection{Prodiges d'Intelligence, éclairer le Monde}\label{subsec:prodigesintelligence}
    \index{Prodiges!d'Intelligence}
    La table ci-dessous récapitule les prodiges présentés dans le livre de base de \gurpstitle{miles Christi}. Pour chaque prodige est indiqué le score en Amour nécessaire ainsi que la page de référence où se trouve sa description.
    \begin{center} 
\tablecaption{\label{t-prodigesintelligence}Prodiges d'Intelligence}
\tablehead
	{\hline \textbf{Prodige} & \textbf{Score} & \textbf{Référence}\\ \hline}
\tabletail
	{\hline \multicolumn{3}{c}{\emph{Suite page suivante...}}\\}
\tablelasttail
	{\hline}
\begin{supertabular}{|c|c|c|}

L'arche de Noé & 1 & \cite[p.245]{MC} \\
Le souffle de la bête & 1 & \cite[p.245]{MC} \\
\rowcolor[gray]{0.8}
Ramener les brebis égarées & 1 & \cite[p.245]{MC} \\ 
\rowcolor[gray]{0.8}
Que la lumière soit & 2 & \cite[p.245]{MC} \\ 
Avant Babel & 2 & \cite[p.246]{MC} \\
A toute parole sa récompense & 2 & \cite[p.246]{MC} \\
\rowcolor[gray]{0.8}
Le songe du berger & 3 & \cite[p.246]{MC} \\ 
\rowcolor[gray]{0.8}
Des yeux pour voir & 3 & \cite[p.246]{MC} \\ 
Trier le bon grain de l'ivraie & 3 & \cite[p.246]{MC} \\
Sonder les coeurs et les reins & 4 & \cite[p.246]{MC} \\
\rowcolor[gray]{0.8}
Les chemins du Seigneur & 4 & \cite[p.246]{MC} \\ 
\rowcolor[gray]{0.8}
Après Babel & 4 & \cite[p.246]{MC} \\ 
Le bâton d'Aaron & 5 & \cite[p.247]{MC} \\
La communion des Justes & 5 & \cite[p.247]{MC} \\
\rowcolor[gray]{0.8}
La marque de la bête & 5 & \cite[p.247]{MC} \\ 
\rowcolor[gray]{0.8}
Urbi et Orbi & 6 & \cite[p.247]{MC} \\ 
Le paraclet & 6 & \cite[p.247]{MC} \\
Dieu le veut & 6 & \cite[p.247]{MC} \\
\rowcolor[gray]{0.8}
Les portes de l'Eden & 7 & \cite[p.248]{MC} \\ 
\rowcolor[gray]{0.8}
Dieu est lumière & 7 & \cite[p.248]{MC} \\ 
La parole du prophète & 7 & \cite[p.248]{MC} \\

\end{supertabular}
\end{center}

\section{Les péchés}\label{sec:peches}
\index{Peches@Péchés}
Chaque fois qu'un frère templier trahit ses voeux monastiques ou qu'il ne se conduit pas en bon chrétien, il commet un péché. Les péchés dans cette adaptation suivent dans l'ensemble les mêmes règles que celle du jeu original. Se reporter à \cite[p.222]{MC} pour en comprendre le fonctionnement. Bien entendu, le suivi des péchés au sein du jeu s'effectue à l'aide de l'Echelle de la Foi\index{Echelle de la Foi} (voir \titleref{sec:echellefoi}, \vpageref{sec:echellefoi}. En revanche, cette adaptation propose différents moyens aux frères templiers de laver leur conscience : ils peuvent s'abîmer en prière, pour libérer une case noircie de leur échelle de la Foi, ils peuvent se confesser, pour en libérer jusqu'à trois en une seule fois, ou ils peuvent subir un châtiment approprié. Ces punitions, décidées par le Maître de Jeu, prennent la forme d'une déroute inattendue, d'une faiblesse physique soudaine, de mots malheureux au pire moment, par exemple. Ils s'apparentent au Paradox de \gurpstitle{Mage : The Ascension} (voir \cite[p.133]{MtA}).
