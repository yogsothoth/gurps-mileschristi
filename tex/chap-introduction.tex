\chapter{Introduction}\label{chap:introduction} \gmc{} est une adaptation du
contexte proposé par le jeu \gurpstitle{miles Christi}{} au système
\gurpstitle{GURPS}. Il ne s'agit pas à proprement parler d'une conversion (bien
que des conseils soient donnés dans ce sens) mais plutôt d'une tentative de
capturer ce qui fait le sel, l'essence de la structure ludique du jeu original
pour transposer ceci dans le système de Steve Jackson Games.

Ainsi, on retrouve le cœur du jeu, à savoir la notion de choix et la
contradiction entre le rôle de moine et celui de chevalier. \gmc, comme
l'original, propose également d'interpréter des personnages particulièrement
puissants, guidés par leur foi et agissant au nom de Dieu.

\gmc{} se distingue en revanche du jeu original en proposant une gestion des
prodiges non pas basée sur une liste de manifestations fermée, mais en laissant
libre cours aux joueurs et au Maître de Jeu dans la description des
manifestations de la volonté divine. De même, l'abandon des personnages par
Dieu, en cas de péché grave, s'inspire du système de gestion du Paradoxe dans
\gurpstitle{Mage: The Ascension} de White Wolf (et le supplément
\gurpstitle{GURPS}{} associé, \gurpstitle{GURPS Mage: The Ascension}{}). Un tel
abandon peut déboucher sur des déconvenues ponctuelles, tel qu'un échec
inattendu lors d'une entreprise quelconque, ou sur une régression générale du
personnage, qui peut voir ses aptitudes guerrières reculer, ou son savoir
devenir confus.
